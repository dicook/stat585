\documentclass{article}
\usepackage{fullpage}

%\usepackage{emerald}
\usepackage{aurical}
\usepackage[T1]{fontenc}

\begin{document}

\title{A Story Composed by Members of Statistics 585, Spring 2012}
% add your name here when you add your part
\author{Di Cook \and Carson \and Mahbubul Majumder \and Dason Kurkiewicz \and Will Landau \and ...}
\maketitle

Once upon a time, deep in the prairies of the Midwest, a farmer named Snedecor planted a new kind of seed. The seeds were dispersed randomly through 3.141593 acres of rich brown Iowan soil. A tiny grey bird, with a bright red spot on both wings, was watching ever so closely  from the stalk of a tall grass close by. The farmer surveyed his spread of seeds, and used his rake to dust  over a covering layer. The sun cast a golden gleam over the field, a contrast against the brilliant blue sky, as the farmer strode away, happy with his work. In darted the grey bird. She landed on a black sod, and proceeded to hop a straight line through the field, sifting the top layer of dirt, eating the seeds. Three perfectly straight lines devoid of seeds resulted. 

The next day, Snedecor convinced his good friend R.A. Fisher to help him estimate the effects of the grey birds actions on his expected harvest. After a careful examination of Snedecor's method of seed dispersion, Fisher reached the conclusion that the seeds were not randomly distributed through the soil. Thus, Fisher had the misfortune of explaining to his friend that any estimation of these effects would be biased. Unhappy with the turn of events, Snedecor swallowed hard and invited his friend inside for some tea and crumpets.

The secretary of Snedecor  noticed both of them getting in having their minds occupied with some complex thoughts. She knew Snedecor very well. But it was this strange R. A. Fisher about whom she did not have much of understanding. Whenever he travelled to this prairies, Snedecor became excited with his various experiments. But today was different, they were not talking to each other. Snedecor prepared tea for both of them and sat on his working table while R. A. Fisher was glaring at the vapor curling from the hot tea. At some point, Fisher told something to Snedecor and both of them rushed out of the room. She knew, none of them had a chance to get a sip from the tea cup before they potentially reached another solution of their disturbing problem. She was 95\% confident that this apparent solution would turn out to be flawed again. 

But his secretary did not know that it was different this time.\\

``I'm sorry I rushed us out of the room so quickly but this just arrived'' said Fisher. 

``What is it?'' Snedecor promptly inquired.

``A note that I think you'll find to be most illuminating'' was Fisher's response before displaying the following to his dashing colleague.


%\ECFSkeetch 
\Fontauri
\Large 
\begin{center}
\fbox{
  \parbox{0.5\linewidth}{
I was the one that sent that bird.
If you want to find out more then
ask 'The Reverend' where to find
me.  He'll know.\\

Love,\\
Evil Zombie Lincoln\\

P.S. I'm holding the Central Limit
Theorem hostage until you find me.
  }
}
\end{center}


\normalfont
\normalsize
\vspace{.5in}

Snedecor didn't understand.  He wondered silently to himself about who this ``Evil Zombie Lincoln'' could be and if he really was an evil undead version of one of the country's greatest Presidents.  Gazing over to his friend Fisher he could see by the look in his eyes that not only did he know of this ``Evil Zombie Lincoln'' but that he has had past dealings with him as well.

Fisher started running and shouted back to his friend ``We need to go see Sir Reverend Bayes as soon as possible.  Not that I'm looking forward to it... The last time I was in a room with him he cut off my right hand.''

% Will Landau: begin contribution

As only the uberest of Mensches have done, the Reverend had long since retired to the floating island of Laputa, the existence of which Jonathan Swift only suspected when he wrote, \underline{Gulliver's Travels}. As Fisher related this secret to Snedecor, they both puzzled over some important practical issues. How would they find the island? How would they breathe once they arrived? After all, the island was currently somewhere in the mesosphere. \newline

The Jackal, eavesdropping, said, ``Not to worry. Laputa has its own atmosphere, which feels like the lower troposphere." \newline

This strange creature's sudden appearance prompted several questions for Snedecor, who immediately voiced his most urgent concern: ``are you going to eat me?" \newline

``No." proudly stated the Jackal. ``I am a Jackal, and therefore I eat bricks." \newline

The Jackal proceeded to eat a lone brick that was concealing the bird who had earlier eaten the seeds. When he was finished, the wondrous extraneous masonry cleaner-upper pointed a deft paw at the feathered rogue and declared, ``You, sir, are a beanstalk." \newline

With a cry of jubilance, glee, passion, enlightenment, morality, inquiry-based learning, and shrill inflections (for it was the bird's lifelong ambition to become a beanstalk), the bird poofed into a beanstalk and grew at an alarming rate, rocketing the two men and the fabulous fanged demigod up into the clouds. With the Central Limit Theorem in captivity, things were no longer as they once seemed...

As the Jackal explained on their way up through the clouds, the Bayes Estimator is always in the form of a beanstalk. That's why the best way to find The Eternal Reverand Bayes is to hop on a beanstalk and wait five minutes. Because although riding a fast-growing beanstalk is extremely risky (you could fall off and die), the Bayes Risk of this undertaking is extremely small because 99.99% of all beanstalks eventually find their way to Bayes. As the Jackal explained all this, Snedecor and Fisher furiously scribbled down notes, particularly the warning that a low-Bayes-Risk undertaking can still be very risky.

% Will Landau: end contribution

\end{document}
